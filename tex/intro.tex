\documentclass[{../apuntes.tex}]{subfiles}
\begin{document}

\chapter{Introducción}

\section*{Contacto con el profesor}

Para contactar con el profesor, puede usarse su e-mail:

\begin{center}
		dirección@de.email
\end{center}

Para concertar una O bien, acudir a las tutorías:

\begin{center}
	Benito Camelas \\
	Despacho 5, 42ª planta
\end{center}

\section*{Evaluación}

La nota de la asignatura consta de dos partes, la parte de los exámenes $E$ y la evaluación contínua $C$.

Los examenes constarán de dos partes:

\begin{itemize}
	\item Un parcial liberatorio de todo lo dado hasta el capítulo 4.
	\item Un examen final con dos partes.
\end{itemize}

La nota de los exámenes será la media de los dos parciales.

La evaluación contínua consistirá en entregables y en controles realizados en el horario de clases.

La expresión de la calificación final de la asignatura será la siguiente:

\[
E_f = \text{máx} \left( E , 0.8 E + 0.2 C \right)
\]

\section*{Temario}

El temario que seguiremos será el siguiente:

\begin{enumerate}
	\item Tema 1: Problemas de contorno
	\item Tema 2: Ondas planas
	\item Tema 3: Guías de ondas
	\item Tema 4: Ecuaciones de Maxwell y potenciales electromagnéticos.
	\item Tema 5: Radiación
	\item Tema 6: Relatividad
\end{enumerate}



\section*{Bibliografía}

Durante el curso no vamos a seguir ningún libro en concreto. Los libros son los siguientes:


\begin{itemize}
	\item Fundamentos de la teoría electromagnética (Reitz, Milford y Christy): Clásico del Electromagnetismo.
	\item Campos Electromagnéticos (Wangness)
	\item Elementos de Electromagnetismo (Matthew Sadiku): Libro muy completo para los capítulos de guias de ondas.
	\item Introduction to Electrodynamics (Griffiths): Bastante recomendable en general. Útil para radiación y relatividad.

\end{itemize}

\end{document}