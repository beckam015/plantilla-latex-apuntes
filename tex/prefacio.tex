\makeatletter
\def\input@path{{../}}
\makeatother

\documentclass[{../cc.tex}]{subfiles}
\begin{document}

\chapter{Prefacio}


Los presentes apuntes son una recopilación de las clases de INTRODUCIR ASIGNATURA del curso 2015/2016 del grado de Ciencias Físicas de la Universidad Complutense de Madrid, impartido por Benito Camelas.

Aunque han sido revisados, es posible que haya alguna errata, ya que los he tomado directamente a \LaTeX en clase. Por otro lado, hay algunas partes que han podido ser complementadas con los libros que aparecen en la presente bibliografía.

\section*{Temario}

El temario que sigue la asignatura, aproximadamente, es el siguiente.

\begin{enumerate}
	\item \textbf{Introducción}. ¿Por qué se necesitan los campos cuánticos relativista? campos bosónicos, fermiónicos, vectoriales y el grupo de Poincaré. Matrices gamma y transformaciones de Lorentz. Espinores zurdos y diestros. Representaciones finitas del grupo de Lorentz.
	\item \textbf{Campos escalares}. Cuantificación canónica del campo escalar libre. propagador de Feynman. teorema de Wick. Campos Escalares en interacción: la fórmula de Gell-Mann-Low para las funciones de Green y el desarrollo en la constante de acoplo.
	\item \textbf{Campos spinoriales}. Cuantificación canónica de un campo espinorial libre. Propagador de Feynman. Teorema de Wick. Campos fermiónicos en interacción: la fórmula de Gell-Mann-Low y el desarrollo en potencias de las constantes de acoplo.
	\item \textbf{Electrodinámica cuántica}. La invariancia gauge $U(1)$ y la cuantificación canónica del campo de un fotón libre. El propagador de Feynamn. El lagrangiano de QED. Las funciones de Green de QED y su desarrollo en potencias de la constante de acoplo.
	\item Matriz $S$ y el formalizmo LSZ. Los procesos de difusión y matriz $S$: secciones eficaces. Cálculo de la sección eficaz del proceso $e^+ e^-$, $\mu^+ \mu^-$ y otros procesos elementales.
	\item \textbf{La integral de camino}. Funciones de correlación e integral de camino para campos escalares. Cuantificación de fermiones e integral de camino. variables de Grassmann.
	\item \textbf{Campos gauge no abelianos}. Nociones elementales de grupos de Lie simples y compactos. campos gauge no abelianos clásicos: la acción de Yang-Mills y su invariancia Gauge. invariancia BRST y cuantificación de campos gauge no abelianos mediante la integral de camino. Fermiones en interacción con campos gauge no abelianos: la integral de camino.
\end{enumerate}

\section*{Bibliografía}

A lo largo del curso no seguimos ningún libro en concreto, sin embargo siempre es recomendable acudir a ellos. Los recomendados por el profesor son los siguientes:


\begin{itemize}
	\item Modern quantum field theory, Cambridge University Press (T. Banks).
	\item Quantum field theory, Westview (M. E. Peskin, D. V. Schroeder).
	\item Quantum field theory, Cambridge University Press (M. Srednicki).
	\item Quantum field theory, Cambride University Press (G. Sterman).

\end{itemize}

\end{document}
